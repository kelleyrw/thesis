% --------------------------------------------------------------------------- %
% --------------------------------------------------------------------------- %
\chapter{Analysis Selections}
\label{ch:evtsel}
% --------------------------------------------------------------------------- %
% --------------------------------------------------------------------------- %
In Chapter~\ref{ch:ss}, we discussed the motivation for selecting same-sign
dilepton pairs. In this Chapter, we discuss the specific selections used to
carry out the analysis. We consider a number of final states characterized
by the number of jets, number of b-tagged jets, scalar sum \pt~of selected
jets (\Ht) and the missing transverse energy (\met). To provide coverage for
a wide-range of generic signatures, we provide separate results for different
lepton \pt~requirements. Specifically, we perform a \hpt~analysis where we
select leptons with a \pt~requirement of 20 \GeV~(\hpt~analysis) as is common
for leptons from $W/Z$ and in many scenarios of stop/sbottom production. We
separately consider dilepton pairs where the \pt~requirement is lowered to 10
\GeV~(\lpt~analysis). This slightly looser selection provides sensitivity to
scenarios with a (partially) compressed spectrum, where at least one lepton
is expected to come from the decay of an off-shell $W/Z$. Finally, we define
search region in bins of \njets, \nbtags, \Ht and \met in order to improve
statistical sensitivity.

This chapter begins with a discussion on the collision and simulated data used
for this thesis. This is followed by the definitions of the selections for
all the physics objects used; and finally, this chapter concludes with the
discussion on the various search regions used in the analysis.

% --------------------------------------------------------------------------- %
% --------------------------------------------------------------------------- %
\section{Data Samples}
\label{sec:evtsel_samples}
% --------------------------------------------------------------------------- %
% --------------------------------------------------------------------------- %

% --------------------------------------------------------------------------- %
\subsection{Collision Data}
\label{sec:evtsel_samples_data}
% --------------------------------------------------------------------------- %
This analysis uses data collected during the 2012 run at $\sqrt{s} = 8\ \TeV$.
Dilepton events are selected from primary datasets collected using electron and
muon trigger paths. Only runs and luminosity sections from good data taking
periods are used, where such periods are defined using the flags delivered by
the CMS data quality monitoring, detector, and physics validation teams. The
datasets used in this analysis are listed in Tables~\ref{tab:evtsel_datasets_hpt},
and \ref{tab:evtsel_datasets_lpt} with the integrated luminosity corresponding to
\usedLumi.

% --------------------------------------------------------------------------- %
\begin{table}[hbt!]
\begin{center}
\caption[Primary datasets used by the \hpt analysis along with the relevant run-ranges]
{\label{tab:evtsel_datasets_hpt}
Primary datasets used by the \hpt analysis along with the relevant run-ranges.
}
\end{center}
\resizebox{0.78\textwidth}{!}{\begin{minipage}{\textwidth}
\begin{tabular}{lc}\hline\hline
Name                                                    & Run Range                          \\ \hline
{\tt /DoubleMu/Run2012A-13Jul2012-v1/AOD              } & 190456-193621                      \\ 
{\tt /DoubleMu/Run2012B-13Jul2012-v1/AOD              } & 193834-196531                      \\ 
{\tt /DoubleMu/Run2012A-recover-06Aug2012-v1/AOD      } & 190949 190945 190906 190895 190782 \\ 
{\tt /DoubleMu/Run2012C-24Aug2012-v1/AOD              } & 198022-198523                      \\ 
{\tt /DoubleMu/Run2012C-PromptReco-v2/AOD             } & 198934-203755                      \\ 
{\tt /DoubleMu/Run2012D-PromptReco-v1/AOD             } & 203773-208913                      \\ 
{\tt /DoubleElectron/Run2012A-13Jul2012-v1/AOD        } & 190456-193621                      \\ 
{\tt /DoubleElectron/Run2012B-13Jul2012-v1/AOD        } & 193834-196531                      \\ 
{\tt /DoubleElectron/Run2012A-recover-06Aug2012-v1/AOD} & 190949 190945 190906 190895 190782 \\ 
{\tt /DoubleElectron/Run2012C-24Aug2012-v1/AOD        } & 198022-198523                      \\ 
{\tt /DoubleElectron/Run2012C-PromptReco-v2/AOD       } & 198934-203755                      \\ 
{\tt /DoubleElectron/Run2012D-PromptReco-v1/AOD       } & 203773-208913                      \\ 
{\tt /MuEG/Run2012A-13Jul2012-v1/AOD                  } & 190456-193621                      \\ 
{\tt /MuEG/Run2012B-13Jul2012-v1/AOD                  } & 193834-196531                      \\ 
{\tt /MuEG/Run2012A-recover-06Aug2012-v1/AOD          } & 190949 190945 190906 190895 190782 \\ 
{\tt /MuEG/Run2012C-24Aug2012-v1/AOD                  } & 198022-198523                      \\ 
{\tt /MuEG/Run2012C-PromptReco-v2/AOD                 } & 198934-203755                      \\ 
{\tt /MuEG/Run2012D-PromptReco-v1/AOD                 } & 203773-208913                      \\ 
\hline\hline
\end{tabular}
\end{minipage} 
}
\end{table}
% --------------------------------------------------------------------------- %
\begin{table}[hbt]
\begin{center}
\caption[Primary datasets used by the \lpt analysis along with the relevant run-ranges]
{\label{tab:evtsel_datasets_lpt}
Primary datasets used by the \lpt analysis along with the relevant run-ranges.
}
\end{center}
\resizebox{0.8\textwidth}{!}{\begin{minipage}{\textwidth}
\begin{tabular}{lc}\hline
Name                                                 & Run Range                          \\ \hline
{\tt /MuHad/Run2012A-13Jul2012-v1/AOD              } & 190456-193621                      \\ 
{\tt /MuHad/Run2012B-13Jul2012-v1/AOD              } & 193834-196531                      \\ 
{\tt /MuHad/Run2012A-recover-06Aug2012-v1/AOD      } & 190949 190945 190906 190895 190782 \\ 
{\tt /MuHad/Run2012C-24Aug2012-v1/AOD              } & 198022-198523                      \\ 
{\tt /MuHad/Run2012C-PromptReco-v2/AOD             } & 198934-203755                      \\ 
{\tt /MuHad/Run2012D-PromptReco-v1/AOD             } & 203773-208913                      \\ 
{\tt /ElectronHad/Run2012A-13Jul2012-v1/AOD        } & 190456-193621                      \\ 
{\tt /ElectronHad/Run2012B-13Jul2012-v1/AOD        } & 193834-196531                      \\ 
{\tt /ElectronHad/Run2012A-recover-06Aug2012-v1/AOD} & 190949 190945 190906 190895 190782 \\ 
{\tt /ElectronHad/Run2012C-24Aug2012-v1/AOD        } & 198022-198523                      \\ 
{\tt /ElectronHad/Run2012C-PromptReco-v2/AOD       } & 198934-203755                      \\ 
{\tt /ElectronHad/Run2012D-PromptReco-v1/AOD       } & 203773-208913                      \\ 
\hline\hline
\end{tabular}
\end{minipage} 
}
\end{table}

% --------------------------------------------------------------------------- %
\subsection{Simulated Data}
\label{sec:evtsel_samples_mc}
% --------------------------------------------------------------------------- %
The analysis uses simulated data samples for candidate signal models and the
relevant backgrounds. The contributions for real same-sign dilepton events are
estimated directly from simulated data. These ``rare'' processes were discussed
in Chapter~\ref{sec:ss_rare} and constitute an irreducible background for
this analysis. Simulated samples of \Zgs, \Wj, and \ttbar processes are used to
cross-check the partially data-driven estimate of the background contribution
from charge mis-reconstruction, as well as systematic uncertainties of
the lepton selection. The data-driven technique developed to estimate the
background from fake leptons is applied to simulated \ttbar and \Wj samples
as part of the systematic uncertainty studies on these methods. Simulated SUSY
samples are used for the study of the systematic uncertainties as well as to
develop a model of the selection efficiencies. These samples are also used
in the extraction of upper limits on the observed and expected cross section
for these models. The contribution from double fakes from QCD processes is
determined using a data-driven method, but the available simulated samples
are used to study some dependencies of the fake lepton prediction method.
Simulated data samples are normalized to an integrated luminosity of \usedLumi,
unless otherwise stated. Details of the simulated samples can be found in
Tables~\ref{tab:evtsel_datasets_rare} and ~\ref{tab:evtsel_datasets_fake}.

% --------------------------------------------------------------------------- %
\begin{table}[!hbt]
\begin{center}
\caption[Sources of true same-sign dileptons from Standard Model processes]
{\label{tab:evtsel_datasets_rare}
Sources of true same-sign dileptons from Standard Model processes. Cross
sections are next-to-leading order. The equivalent integrated luminosity
of these simulated events is listed in the column on the far right. The
contributions from these process to the background is taken directly from
simulation.
}
\footnotesize{
    \begin{tabular}{lccc}
    \hline\hline
    Sample     & Cross Section (pb) & Equivalent Luminosity (\fbinv) \\ \hline
    \ttZ       & 0.208              & 1021                           \\
    \ttW       & 0.232              & 845                            \\
    \ttWW      & 0.00204            & 106931                         \\
    \ttG       & 2.17               & 33                             \\
    \tbZ       & 0.0114             & 13026                          \\
    \ZZZ       & 0.00554            & 40692                          \\
    \WWW       & 0.0822             & 2737                           \\
    \WWG       & 0.528              & 407                            \\
    \WZZ       & 0.0192             & 12946                          \\
    \WWZ       & 0.0580             & 3832                           \\
    \ZZ        & 0.177              & 27177                          \\
    \WZ        & 1.058              & 1908                           \\
    \qqWmWm    & 0.0889             & 1084                           \\
    \qqWpWp    & 0.248              & 402                            \\
    \WWdps     & 0.588              & 1418                           \\
    \Wgsmm     & 1.91               & 156                            \\
    \Wgstt     & 0.336              & 148                            \\
    \HToWW     & 0.260              & 769                            \\
    \HToZZ     & 0.0320             & 15652                          \\
    \HToTauTau & 0.0177             & 5478                           \\
    \hline\hline
    \end{tabular}
}
\end{center}
\end{table}
% --------------------------------------------------------------------------- %

% --------------------------------------------------------------------------- %
\begin{table}[!hbt]
\begin{center}
\caption[Sources of ``false`` same-sign dileptons from Standard Model processes]
{\label{tab:evtsel_datasets_fake}
Sources of ``fake`` same-sign dileptons from Standard Model processes. Cross
sections are next-to-leading order. The equivalent integrated luminosity
of these simulated events is listed in the column on the far right. The
contributions from these process to the background is taken directly from
simulation.
}
\footnotesize{
    \begin{tabular}{lccc}
    \hline\hline
    Sample                    & Cross Section (pb) & Equivalent Luminosity (\fbinv) \\ \hline
    \ttbar                    & 225                & 30                             \\
    \ttdil                    & 25                 & 493                            \\
    \ttslq                    & 103                & 248                            \\
    \tthad                    & 107                & 292                            \\
    $t$, s-channel            & 3.89               & 69                             \\
    $\overline{t}$, s-channel & 1.76               & 80                             \\
    $t$, s-channel            & 55.5               & 66                             \\
    $\overline{t}$, s-channel & 30.0               & 63                             \\
    $tW$                      & 11.2               & 45                             \\
    $overline{t}W$            & 11.2               & 44                             \\
    \Zgll                     & 3533               & 8.62                           \\
    \Wj                       & 37509              & varies                         \\
    \WW                       & 5.81               & 332                            \\
    \hline\hline
    \end{tabular}
}
\end{center}
\end{table}
% --------------------------------------------------------------------------- %

% --------------------------------------------------------------------------- %
\section{Event Selections}
\label{sec:evtsel_evt}
% --------------------------------------------------------------------------- %
% --------------------------------------------------------------------------- %
The event selection consists of the following general requirements with
amplifying details in the following sub-sections:
\begin{itemize}
\item we require a dilepton trigger;
\item we select events with two good same sign isolated leptons; the \hpt
analysis requires both leptons to have $\pt > 20~\GeV$ and the \lpt analysis
requires the leptons to have $\pt > 10~\GeV$.
\item $\m_{\ell\ell} > 8 \GeV$ to reject low mass resonances.
\item For most search regions, a significant amount of \met to reduce SM
backgrounds, particularly that from charge mis-measurement in \Zll events.
\item We require a significant amount of hadronic energy to reduce standard
model backgrounds, particularly mi-identified leptons in \Wj events.
\item We veto events if the invariant mass of either hypothesis lepton and a
third good lepton in the event is consistent with the Z mass. This requirement
is chosen to suppress the irreducible background from \WZ and \ZZ events.
\item We veto events if the invariant mass of either hypothesis lepton and a
third good lepton in the event is $< 12~\GeV$. This requirement is chosen to
suppress the irreducible background from \gs and low mass resonances.
\end{itemize}

To reject known machine background resulting in high activity in the pixel
layers, we require that no fewer than 25\% of the tracks in event are high
purity in events having 10 or more tracks. Also, to ensure that there was a
reconstructed collision, we require that at least on one good primary vertex
(PV). A good PV is selected by requiring:
\begin{itemize}
\item the PV is not considered a fake,
\item the number of degrees of freedom > 4,
\item $|\rho| < 2~\cm$,
\item $|z| < 24~\cm$.
\end{itemize}

% --------------------------------------------------------------------------- %
\subsection{Trigger Selection}
\label{sec:evtsel_trig}
% --------------------------------------------------------------------------- %
We use a handful of double lepton trigger paths to select events in data. An
event in the \ee final state is required to pass the relevant double electron
trigger, a \mm event must pass the double muon trigger and an \em event is
required to pass one of the electron-muon cross triggers. Because of the
rapidly changing trigger menu, many trigger paths were not implemented in the
simulation and thus no trigger requirement is made here. Instead, as discussed
in Section~\ref{sec:eff_trig}, a weight is applied to each event, based on the
trigger efficiencies measured from data. A list of trigger paths used to select
signal-like events can be found in in the following table:

% --------------------------------------------------------------------------- %
\begin{table}[!hbt]
\begin{center}
\caption[The trigger paths used for the \hpt and \lpt analyses]
{\label{tab:evtsel_trig}
The trigger paths used for the \hpt and \lpt analyses. Note that for the \hpt
triggers, {\tt XXLstr} stands for {\tt CaloIdT\_CaloIsoVL\_TrkIdVL\_TrkIsoVL}
which is explained in the text below. For the \lpt triggers, {\tt PFHT175} refers to
particle flow based $\Ht > 175 \GeV$.
}
\begin{tabular}{l|c|c}
\hline\hline
Analysis Type         & Channel & Trigger                                                   \\ \hline
\multirow{3}{*}{\hpt} & \ee & {\tt HLT\_Ele17\_XXLstr\_Ele8\_XXLstr}                        \\
                      & \em & {\tt HLT\_Mu17\_Ele8\_XXLstr} or {\tt HLT\_Mu8\_Ele17\_XXLstr}\\
                      & \mm & {\tt HLT\_Mu17\_Mu8}                                          \\ \hline
\multirow{3}{*}{\lpt} & \ee & {\tt HLT\_DoubleEle8\_CaloIdT\_TrkIdVL\_Mass8\_PFHT175}       \\
                      & \em & {\tt HLT\_Mu8\_Ele8\_CaloIdT\_TrkIdVL\_Mass8\_PFHT175}        \\
                      & \mm & {\tt HLT\_DoubleMu8\_Mass8\_PFHT175}                          \\
\hline\hline
\end{tabular}
\end{center}
\end{table}

% --------------------------------------------------------------------------- %
The \hpt analysis select event with two leptons having $\pt > 20\ \GeV$.
Dielectron events in the \hpt triggers are selected using triggers with
\Et requirements of 17 and 8 \GeV on the two legs. Triggers with electron
legs impose online calorimeter and tracker identification and isolating
requirements. The online selections are sufficiently loose compared to the
analysis isolation selection that little inefficiency is introduced. However,
care must be taken in the data-driven background estimate so that a bias is not
introduced. This issue was studied in detail elsewhere and found to not be a
significant concern~\cite{an_ufl2013}. The electron-muon cross triggers have
muon \pt requirements of 17 \GeV (8 \GeV) and corresponding \Et requirements of
8 \GeV (17 \GeV). The dimuon triggers have only a \pt requirement of
17 and 8 \GeV on the two legs.

The trigger paths used for the \lpt analyses have lower \pt
requirements due to the lower threshold on the lepton (i.e.~$\pt > 10\ \GeV$);
however, they impose an additional \HT requirement to keep the rate within the
allotted bandwidth. The online \HT requirement was 175 \GeV using particle flow
based jets. Later runs also imposed a pile-up correcting to the \HT calculation.
All three channels use triggers that also impose a dilepton invariant mass of
8 \GeV. The dielectron triggers have an \Et requirement of 8 \GeV on both
lets. Unlike the \hpt triggers, the electron legs impose only calorimeter
and tracker identification requirements -- no isolation requirement. The
electron-muon cross triggers require the same identification requirements on
the electron leg and select electrons with $\Et > 8\ \GeV$ and muons with $\pt
> 8\ GeV$. The \pt requirement in the dimuon trigger is 8 \GeV on both legs.

% --------------------------------------------------------------------------- %
\subsection{Muon Selection}
\label{sec:evtsel_mu}
% --------------------------------------------------------------------------- %
Muon reconstruction was discussed previously in Section~\ref{sec:cms_muon}. To reject
fakes with relatively little loss in efficiency, we require muons to be
reconstructed as both global muons and particle flow muons. To further
reject fakes and poorly reconstructed muons we apply the following quality
requirements~\cite{an_ssb2013,muonidtwiki}:
% --------------------------------------------------------------------------- %
\begin{itemize}
\item We require the muon to have $\pt > 20\ \GeV$ and $\pt > 10\ \GeV$ for the
\hpt and \lpt analysis, respectively.
\item We require the muon to have $|\eta| < 2.4$ which is the edge of the
muon detector coverage. Due to the finite luminous region, some muons can
have a track with $|\eta| > 2.4$; however, we do not consider these as the
inefficiency is relatively large.
\item We require that the global fit of the muon have $\chi^2/\rm{ndof} < 10$
to reject poorly reconstructed muons.
\item We require that at least one muon sub-detector hit is used in the global
fit.
\item We require that at least one hit is in the pixel layers.
\item We require that the muon have at least 6 layers in the silicon tracker,
to reject poorly reconstructed muons as well as muons originating late in the
tracker from decays in flight.
\item We require that the energy deposits in the veto cones are less that
4 \GeV in the ECAL and 6 \GeV in the HCAL. Veto deposits are calculated in
cones of size $\DR = 0.07,\ 0.1$ for the ECAL and HCAL, respectively. Vetoing
on the energy in the cones rejects fake muons produced when hadrons punch
through the calorimeter into the muon system. The requirement can introduce
an inefficiency for muons with \hpt or in a busy event environment such as a
\ttbar-like event~\cite{an_098_2008}.
\item We require the muon to have at least two segments in the muon chambers,
to reject reconstructed objects that are mis-identified as muons because they
punched through the calorimeter and interacted in the first layer of the
muons system (e.g. from kaons).
\item We require that the muon have transverse impact parameter (\dzero) $< 50\
\um$, where the \dzero has been calculated with respect to the primary vertex.
A tight selection reduces background muons from the decay of b-hadrons, as well
as decays in flight.
\item We require that the inner track $z$ be less than 1 mm from the first good
vertex, which we take to be the event vertex. This requirement helps reject
mis-reconstructed muons as well as those originating from pile-up interactions.
\end{itemize}
% --------------------------------------------------------------------------- %
The isolation follows the POG recommendation, using particle-flow based        
isolation with a $\Delta\beta$ correction for PU. However, a smaller cone      
size of \DR\ $=$ 0.3 is adopted due to the high hadronic activity expected in  
signal-like events. The isolation is calculated using                             
% --------------------------------------------------------------------------- %
$$
\relIso = [ \Sigma_{\rm ch} + {\rm max}(0, \Sigma_{\rm nh} + \Sigma_{\rm ph} - 0.5 \Delta\beta) ]/\pt,
$$
% --------------------------------------------------------------------------- %
where $\Sigma_{\rm ch, nh, ph}$ are the sums of the \pt of the charged hadron,
neutral hadron, and photon particle flow candidates, respectively. Here the
charged hadrons are matched to the PV and a 0.5~\GeV threshold is applied
on neutral hadrons and photons. The $\Delta\beta$ correction is determined
from the sum \pt\ of charged hadrons not matched to the PV with a threshold of
0.5~\GeV\ in a cone of the same size as the isolation. The \relIso is required
to be less than 0.1~\cite{muonpfisotwiki}.
% --------------------------------------------------------------------------- %
\subsection{Electron Selection}
\label{sec:evtsel_el}
% --------------------------------------------------------------------------- %
Electrons in this analysis are reconstructed using the Gaussian Sum Filter
(GSF) algorithm as was discussed in Section~\ref{sec:cms_electron}. For this analysis we
require the GSF electrons to passed the additional requirements:
% --------------------------------------------------------------------------- %
\begin{itemize}
\item We require the electron to have $\pt > 20\ \GeV$ and $\pt > 10\ \GeV$ for
the \hpt and \lpt analyses, respectively.
\item We require the electron to have $|\eta| < 2.4$.
\item We require the number of missing expected inner hits to be zero.
\item We require the $H/E < 0.1\ (0.075)$ in the barrel (endcap) to match the
requirements on the trigger.
\item We require that the electron have a $|\dzero| < 100\ \um$, where the
\dzero has been calculated with respect to the primary vertex. A tight
selection on the impact parameter reduces background from fake electrons and
electrons from photon conversions in the tracker.
\item We require that there is no muon within a cone of $\DR = 0.1$ about the
electron. Only muons passing the selection in Section~\ref{sec:evtsel_mu} are
considered. This veto rejects the situation where a muon is mis-reconstructed
as an electron.
\item We require the electron to pass the VBTF80 identification (see
Table~\ref{tab:evtsel_elid}).
\item We reject electrons with a supercluster \aeta in the transition region
between the barrel and endcap of the ECAL ($1.4442 < |\eta_{SC}| < 1.566$).
Electrons in this region are poorly reconstructed.
\item We require that the GSF track $z$ be less than 1 mm from the first
good vertex, which we take to be the event vertex. This requirement helps
reject mis-reconstructed electrons as well as those originating from pile-up
interactions.
\item We require that all three charge measurements for an electron agree.
One charge comes from the curvature of the GSF track. A second comes form the
curvature of the associated CTF track. We require all electrons to have an
associated CTF track. The last charge, the so-called supercluster charge, is
determined using the relative position of the supercluster with respect to the
projected track from the pixel seed.
\item To reject conversions, we apply a veto of a good reconstructed conversion
vertex. A conversion vertex is considered good if it has no tracker hits
towards the beam, has a fit probability above $10^{-6}$, has a displacement of
more than 2 \cm, and the CTF track matching to the electron should be a part
of the conversion vertex. No requirement is made on the vertex quality flag
corresponding to merging and arbitration.
\end{itemize}
% --------------------------------------------------------------------------- %
The isolation follows the POG recommendation, using particle-flow based
isolation with a cone size of $\DR = 0.3$~\cite{egammapfisotwiki}. In the
endcap, an inner veto of \DR\ $=$ 0.015 (0.08) is imposed for charged hadrons
(photons). The isolation is corrected for PU by subtracting from the neutral
isolation components a term defined by the product of the average event energy
density and the effective area of the isolation cone~\cite{egammaisorhoaeff}.
The neutral component after correction is required to be non-negative. The
isolation relative to the electron \pt\ is required to be less than 0.09.

The VBTF80 identification uses the selection from Table~\ref{tab:evtsel_elid}.
CMS defines the barrel/endcap transition to be at $|\eta_{SC}| = 1.479$.
The selection values are chosen to have approximately 80\% efficiency for
the electrons from W and Z decays. The values reported are from CMS
recommended values~\cite{egammaidtwiki}.
% --------------------------------------------------------------------------- %

\begin{table}[!hbt]
\begin{center}
\caption[Details of the modified VBTF80 electron identification]
{\label{tab:evtsel_elid}
Details of the modified VBTF80 electron identification.
}
\begin{tabular}{l|c|c}
\hline\hline
Selection Variable & Selection Value (barrel) & Selection Value (endcap) \\ \hline
\sieie             & < 0.01                   & < 0.03                   \\ 
$\dphi_{in}$       & < 0.06                   & < 0.03                   \\ 
$\eta{in}$         & < 0.004                  & < 0.007                  \\ 
$H/E$              & < 0.1                    & < 0.075                  \\ \cline{2-3}
$(1/E - 1/P)$      & \multicolumn{2}{c}{> 0.05}                          \\ 
\hline\hline
\end{tabular}
\end{center}
\end{table}
% --------------------------------------------------------------------------- %
The \sieie variable cuts on the energy deposits in the ECAL, electrons have
a characteristic shower shape that is narrow in $\eta$ but wide in
$\phi$, the latter due to the bremsstrahlung in the tracker of the electrons.
The $\dphi_{in}$ and $\deta_{int}$ variables are cut on the track-to-cluster
matching. The final variable, $H/E$, is calculated as a ratio of energies in
the ECAL and HCAL veto cones. An electron is expected to deposit nearly all of
its energy in the ECAL, while a hadron usually deposits significant energy in
the HCAL, as well.

% --------------------------------------------------------------------------- %
\subsection{Lepton Pair Disambiguation}
\label{sec:evtsel_pair}
% --------------------------------------------------------------------------- %
In events with multiple same-sign lepton pairs passing the selection above,
only one pair is selected according to the following prescription:
\begin{itemize}
\item We give preference to \mm pairs over \em, which are chosen over \ee pairs.
\item If multiple candidates remain, the pair with the highest scalar sum \pt is chosen.  
\end{itemize}

% --------------------------------------------------------------------------- %
\subsection{Jet Selection}
\label{sec:evtsel_jets}
% --------------------------------------------------------------------------- %
We require the presence of energetic hadronic activity in the event as new
physics with a large cross section is expected to be produced via the strong
interaction. We choose to use jets built from particle-flow candidates, as
they provide the best scale and resolution for jets typical of those produced
in SM process such as \ttbar events. Particle-flow candidate reconstruction
was discussed in section Section~\ref{sec:cms_pf} and jet reconstruction was
discussed in Section~\ref{sec:cms_jets}.
% --------------------------------------------------------------------------- %

\begin{table}[!hbt]
\begin{center}
\caption[Details of the loose particle flow identification]
{\label{tab:evtsel_jetid}
Details of the loose particle flow identification. From~\cite{an_003_2010}.
}
\begin{tabular}{l|c|c}
\hline\hline
Selection Variable                           & Selection Value & Comment       \\ \hline
fraction of energy from neutral hadrons      & < 0.99          &               \\ 
fraction of energy from neutral EM particles & < 0.99          &               \\ 
number of particle flow candidates           & > 1             &               \\ 
fraction of energy from charged hadrons      & > 0             & $\aeta < 2.4$ \\ 
fraction of energy from charged EM particles & < 0.99          & $\aeta < 2.4$ \\ 
number of charged particle flow candidates   & > 0             & $\aeta < 2.4$ \\ 
\hline\hline
\end{tabular}
\end{center}
\end{table}
% --------------------------------------------------------------------------- %
Jets are reconstructed with the anti-$k_T$ algorithm with parameter $R =
0.5$. Jets in simulation have L1FastJetL2L3 corrections applied, as well as
jets in data have L2L3 residual corrections applied~\cite{jetcorrtwiki}.
Selected jets are required to pass the loose particle flow jet ID described
in Table~\ref{tab:evtsel_jetid}. These selections are very loose and are
primarily intended to reject jets that are unambiguously due to detector
noise. The inefficiency of this selection on simulation is less than
1\%~\cite{an_003_2010}. Additionally, jets are required to be separated by $\DR
> 0.4$ from any hypothesis leptons and other leptons passing the selection
above.

In the event, there must be at least two particle-flow jets with $\pt > 40\
\GeV$ and $\aeta < 2.4$. The fiducial cut coincides with the extent of the
tracker. In addition to counting jets, the analysis also makes a selection
based on the total hadronic activity in the event, \Ht. The \Ht is calculated
as the scalar sum of the \pt of all jets passing the jet counting criteria
discussed above. Even a modest \Ht requirement suppresses the background from
failures of lepton identification in \Wj events. Having two jets in the event
imposes a minimal \Ht requirement of 80 \GeV.

% --------------------------------------------------------------------------- %
\subsection{B-Tagging Selection}
\label{sec:evtsel_btag}
% --------------------------------------------------------------------------- %
Jets selected based on the requirements of Section~\ref{sec:evtsel_jets} are
b-tagged using the Combined Secondary Vertex method with the medium working
point (CSVM) tagger. This tagger identifies jets with discriminant larger
than 0.679 as b-tagged. The details of b-jet identification was discussed in
Section~\ref{sec:cms_btags}.

% --------------------------------------------------------------------------- %
\subsection{Missing Transverse Energy Selection}
\label{sec:evtsel_met}
% --------------------------------------------------------------------------- %
Missing transverse energy is a natural requirement for new physics searches.
Many models contain weakly interacting particles. For example, some of the
SUSY models considered produce leptons via decay chains ending in a lightest,
non-interacting particle (LSP). CMS uses three different \met reconstruction
algorithms; however, for this analysis, we use \met reconstructed from a
vector sum of the particle flow candidates which was discussed in detail in
Section\ref{sec:cms_met}. For almost all of the search scenarios, we impose a
minimum \met requirement of at least 30 \GeV. Event a modest \met requirement
is effective at suppressing background from Drell-Yan process in the \ee and
\em final states where the charge on one of the final state electrons is
mis-reconstructed.

% --------------------------------------------------------------------------- %
\subsection{Z and $\gamma$* veto}
%[Z and Gamma* veto]
\label{sec:evtsel_zveto}
% --------------------------------------------------------------------------- %
One of the primary irreducible backgrounds to the same-sign dilepton search
comes from \WZ and \ZZ production, where the bosons both decay to leptons.
A natural same sign hypothesis is formed using a lepton from each of the
two bosons. In the case of the \WZ, the lepton from the $W$ comes together
with a neutrino of the same flavor providing a natural source of \met. To
reduce this background, we reject events for which one of the hypothesis
leptons and a third lepton in the event have an invariant mass consistent
with the $Z$, defined to be between 76 and 106 \GeV. Along the same lines, to
reject events with a virtual photon, we also require the invariant mass to
be less than 12 \GeV. We require the third lepton to be the same flavor and
opposite sign and to pass the identification and isolation criteria described
in Sections~\ref{sec:evtsel_el} and~\ref{sec:evtsel_mu}.

% --------------------------------------------------------------------------- %
% --------------------------------------------------------------------------- %
\section{Search Regions}
\label{sec:evtsel_sr}
% --------------------------------------------------------------------------- %
% --------------------------------------------------------------------------- %
As in previous versions of this
analysis~\cite{an_ssb2011,an_ssb2012,an_ssb2012hcp}, there is a minimal
baseline selection asking for a same-sign lepton pair and at least two jets
(or b-tagged jets). Here, we define separately a control region (baseline)
selection for each of the two lepton selections (\hpt and \lpt) in the
following b-tagged categories:
% --------------------------------------------------------------------------- %
\begin{itemize}
\item no \bj requirement,
\item require exactly one \bj,
\item require two or more \bjs.
\end{itemize}
% --------------------------------------------------------------------------- %
In the baseline regions, a relatively low \met requirement is made in an
effort to be inclusive to a wide variety of signatures such as same-sign top
productions or models with R-parity violation. Specifically, for R-parity
violating models we expect events that have many high \pt jets. Thus a
relatively high $\Ht>500\ \GeV$ requirement is required when searching for
this signature and any \met requirement is relaxed. Many of the viable models,
however, involve leptons arising from $W$ decays and thus naturally have \met
from the accompanying neutrino. To improve sensitivity to these signatures,
a region with $\met > 30 \GeV$ is introduced when $\Ht<500\ \GeV$. Table
\ref{tab:evtsel_srbl} summarizes the baseline search regions used in this
analysis where we define baselines for each of the analysis types. There are
a total of six baseline regions. For convenience, each search region is a
assigned a number, given in the far right column.

% --------------------------------------------------------------------------- %
\begin{table}[htb!]
\begin{center}
\caption[Summary of the baseline search regions considered in the \hpt and \lpt analysis]
{\label{tab:evtsel_srbl}
Summary of the baseline search regions considered in the \hpt and \lpt analysis.
}
\end{center}
\resizebox{0.7\textwidth}{!}{\begin{minipage}{\textwidth}
\begin{tabular}{c|c|c|c|c|c|c}
\hline\hline
Analysis              & min lepton \pt ($\mu$, e) (\GeV) & \Ht (\GeV)           & \met (\GeV)                               & \njets             & \nbtags & Search Region \# \\ \hline
\multirow{3}{*}{\hpt} & \multirow{3}{*}{20, 20}          & \multirow{3}{*}{80}  & \multirow{3}{*}{30 if $\Ht < 500$ else 0} & \multirow{3}{*}{2} & $\ge$ 0 & SR0              \\ \cline{6-7}
                      &                                  &                      &                                           &                    & = 1     & SR10             \\ \cline{6-7}
                      &                                  &                      &                                           &                    & $\ge$ 2 & SR20             \\ \hline
\multirow{3}{*}{\lpt} & \multirow{3}{*}{10, 10}          & \multirow{3}{*}{250} & \multirow{3}{*}{30 if $\Ht < 500$ else 0} & \multirow{3}{*}{2} & $\ge$ 0 & SR0              \\ \cline{6-7}
                      &                                  &                      &                                           &                    & = 1     & SR10             \\ \cline{6-7}
                      &                                  &                      &                                           &                    & $\ge$ 2 & SR20             \\ \hline
\end{tabular}
\end{minipage}
}
\end{table}  

% --------------------------------------------------------------------------- %
In contrast from the fall analysis, this analysis is performed as a
simultaneous counting experiment in multiple independent bins or search
regions. The search regions are defined by event selections on the lepton
\pt, \Ht, \met, \njets, and \nbtags. To improve sensitivity to SUSY scenarios
involving top and bottom squark production, we define regions with tighter \Ht
and \met requirements. The \met-\Ht plane is divided into 4 major regions:
% --------------------------------------------------------------------------- %
\begin{itemize}
	\item Low-\Ht~Low-\met~region: $200\ \GeV < \Ht < 400\ \GeV$, $50\ \GeV < \met < 120\ \GeV$.
	\item Low-\Ht~high-\met~region: $200\ \GeV < \Ht < 400\ \GeV$, $\met > 120\ \GeV$.
	\item high-\Ht~Low-\met~region: $\Ht > 400\ \GeV$, $50\ \GeV < \met < 120\ \GeV$.
	\item High-\Ht~high-\met~region: $\Ht > 400\ \GeV$, $\met > 120\ \GeV$.
\end{itemize}

% --------------------------------------------------------------------------- %
As will be discussed in Section~\ref{sec:results_int} all of the SUSY scenarios that we
consider explicitly have at least two quarks, zero to four b-quarks, up to two
hadronically decaying W bosons, and at least two neutrinos and two LSPs. Thus,
in all search regions, we require at least 2 jets but the \njets requirement is
broken into two categories:
% --------------------------------------------------------------------------- %
\begin{itemize}
\item two to three jets
\item four or more jets 
\end{itemize}
% --------------------------------------------------------------------------- %
Additionally, the SUSY inspired search regions also have exclusively defined
values on the \nbtags with exactly zero, exactly one and two or more b-tagged
jets.
% --------------------------------------------------------------------------- %
The search regions for the \hpt analysis are summarized in
Table~\ref{tab:evtsel_sr_hpt}.

\begin{table}[!htb]
\begin{center}
\caption[Search regions selected for the high \pt analysis]
{\label{tab:evtsel_sr_hpt}
Search regions selected for the high \pt search where we require $\Ht>200\ \GeV$.
}
\begin{tabular}{c|c|c|c|c}
\hline\hline
\nbtags                   & \met                    & \njets   & \Ht[200-400] & \Ht[$>400$] \\ \hline
\multirow{4}{*}{$=0$}     & \multirow{2}{*}{50-120} & 2-3      & SR1          & SR2         \\ \cline{3-5}
                          &                         & $\geq 4$ & SR3          & SR4         \\ \cline{2-5}
                          & \multirow{2}{*}{$>120$} & 2-3      & SR5          & SR6         \\ \cline{3-5}
                          &                         & $\geq 4$ & SR7          & SR8         \\ \hline
\multirow{4}{*}{$=1$}     & \multirow{2}{*}{50-120} & 2-3      & SR11         & SR12        \\ \cline{3-5}
                          &                         & $\geq 4$ & SR13         & SR14        \\ \cline{2-5}
                          & \multirow{2}{*}{$>120$} & 2-3      & SR15         & SR16        \\ \cline{3-5}
                          &                         & $\geq 4$ & SR17         & SR18        \\ \hline
\multirow{4}{*}{$\geq 2$} & \multirow{2}{*}{50-120} & 2-3      & SR21         & SR22        \\ \cline{3-5}
                          &                         & $\geq 4$ & SR23         & SR24        \\ \cline{2-5}
                          & \multirow{2}{*}{$>120$} & 2-3      & SR25         & SR26        \\ \cline{3-5}
                          &                         & $\geq 4$ & SR27         & SR28        \\ \hline\hline
\end{tabular}
\end{center}
\end{table}

% --------------------------------------------------------------------------- %
The search regions for the \lpt analysis are essentially the same as the
\hpt analysis with one change. In order to avoid the threshold effect from
the $\Ht>175\ \GeV$ requirement from the triggers used in \lpt analysis, the
minimum \Ht requirement is raised to 250 \GeV. The search regions for the \lpt
analysis are summarized in Table~\ref{tab:evtsel_sr_lpt}.
% --------------------------------------------------------------------------- %
\begin{table}[!htb]
\begin{center}
\caption[Search regions selected for the low \pt~analysis]
{\label{tab:evtsel_sr_lpt}
Search regions selected for the \lpt search where we require
$\Ht>250\ \GeV$.
}
\begin{tabular}{c|c|c|c|c}
\hline\hline
\nbtags                   & \met                    & \njets   & \Ht[250-400] & \Ht[$>400$] \\ \hline
\multirow{4}{*}{$=0$}     & \multirow{2}{*}{50-120} & 2-3      & SR1          & SR2         \\ \cline{3-5}
                          &                         & $\geq 4$ & SR3          & SR4         \\ \cline{2-5}
                          & \multirow{2}{*}{$>120$} & 2-3      & SR5          & SR6         \\ \cline{3-5}
                          &                         & $\geq 4$ & SR7          & SR8         \\ \hline
\multirow{4}{*}{$=1$}     & \multirow{2}{*}{50-120} & 2-3      & SR11         & SR12        \\ \cline{3-5}
                          &                         & $\geq 4$ & SR13         & SR14        \\ \cline{2-5}
                          & \multirow{2}{*}{$>120$} & 2-3      & SR15         & SR16        \\ \cline{3-5}
                          &                         & $\geq 4$ & SR17         & SR18        \\ \hline
\multirow{4}{*}{$\geq 2$} & \multirow{2}{*}{50-120} & 2-3      & SR21         & SR22        \\ \cline{3-5}
                          &                         & $\geq 4$ & SR23         & SR24        \\ \cline{2-5}
                          & \multirow{2}{*}{$>120$} & 2-3      & SR25         & SR26        \\ \cline{3-5}
                          &                         & $\geq 4$ & SR27         & SR28        \\ \hline\hline
\end{tabular}
\end{center}
\end{table}  

% --------------------------------------------------------------------------- %
