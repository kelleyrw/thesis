% --------------------------------------------------------------------------- %
% --------------------------------------------------------------------------- %
\chapter{Summary and Conclusions}
\label {ch:conclusion}
% --------------------------------------------------------------------------- %
% --------------------------------------------------------------------------- %
This thesis reports on a search for new physics in a final state with two
same-sign leptons, missing transverse energy, and significant hadronic
activity. The results use the data collected by the CMS detector from 2012
corresponding to 19.5 \fbinv of integrated luminosity. No significant deviation
from the Standard Model expectations were observed.

The results are used to set upper limits on the same-sign top-pair production
cross section $\sigma(pp \to tt + {\overline{t}}{\overline{t}}) < 0.71~\pb$ and
$\sigma(pp \to tt) < 0.33~\pb$ at the 95\% confidence level. Also the upper
limit on the SM cross section for the four-top-quark production is computed to
be $\sigma(pp \to tt\overline{t}\overline{t}) < 47~\fb$ at the 95\% confidence level.

Upper limits on the cross section for several models of supersymmetry are also
set. Specifically, sparticle production cross-sections calculated assuming that
gluinos decay exclusively into top or bottom squarks. The probed models show an
upper limit on gluinos masses up to $\sim 1050~\GeVcc$ and on bottom squarks
with masses up to $~\sim 500~\GeVcc$. In models where no third generation
squarks were involved in the gluino decays, a somewhat weaker limit is obtained
for the probed gluino masses of up to $\sim 900~\GeVcc$.

The future of this analysis looks towards the next LHC run that will increase
the center-of-mass energy to 13 or 14 \TeV. This, along with improvements
to the estimation for the main SM backgrounds (i.e.~\ttW,~\ttZ,~\qqWW,
and~\HToTauTau) will result in a large increase in the sensitivity of this
analysis for all of the supersymmetry models probed. The CMS collaboration
intends to pursue this analysis when the LHC comes back online sometime in
2015.
