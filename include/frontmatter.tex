% No symbols, formulas, superscripts, or Greek letters are allowed
% in your title.
\title{Search for New Physics in Proton-Proton Collisions at 8 TeV Center of Mass Energy with a Final State of Same-Sign Dileptons and Jets}

\author{Ryan Ward Kelley}
\degreeyear{2013}

% Master's Degree theses will NOT be formatted properly with this file.
\degreetitle{Doctor of Philosophy} 

\field{Physics}
\chair{Professor Avraham Yagil}

%  The rest of the committee members  must be alphabetized by last name.
\othermembers{
Professor Claudio Campagnari\\ 
Professor Robert Continetti\\
Professor Aneesh Manohar\\
Professor Frank W\"urthwein\\
}
\numberofmembers{5} % |chair| + |cochair| + |othermembers|


\begin{frontmatter}
\makefrontmatter                                                               

%% ----------------------------------------------------------------------- %%
%% DEDICATION
%% ----------------------------------------------------------------------- %%

\begin{dedication}                                                             
  I dedicate this thesis to Jennifer, my beautiful wife, for her endless patience, love, and support.  We did it -- Swipe swipe!
%    Add dedication
\end{dedication}                                                               
\clearpage 

%% ----------------------------------------------------------------------- %%
%% EPIGRAPH
%% ----------------------------------------------------------------------- %%

%  The same choices that applied to the dedication apply here.
% \begin{epigraph} % The style file will position the text for you.              
%   \it{Mon seul d\'esir est de m'enrichir de nouvelles pens\'ees exaltantes.} \\
%   ---Ren\'e Magritte
% \end{epigraph}                                                                 
\begin{myepigraph} % You position the text yourself.                           
  \vfil                                                                        
  \vfil 
  \hfill {\it perfer et obdura; dolor hic tibi proderit olim.} \\
  \vfil 
  \noindent {\it Be patient and tough; some day this pain will be useful to you.} \hfill \\
  \vfil 
  \hfill ---old latin saying
  \vfil 
\end{myepigraph}                                                               

\tableofcontents
\listoffigures  % Uncomment if you have any Figures                            
\listoftables   % Uncomment if you have any Tables                             

\begin{acknowledgements}                                                       
% Acknowledgements\ldots
Particle physics turned out to be harder that I expected. It's a huge
undertaking with many people and a huge amount of technical
knowledge to absorb. This does not include actually learning particle physics
itself. There were many times where I didn't think I would make it to this
point and without the support of my friends, family, and collogues, I wouldn't
have. Thank you!

To make this possible, I must thank my little brother Mike. He helped me
relearn and expand my programming skills to a level I didn't even know existed.
His willingness to get dirty with me and talk out all sorts of design and
technical issues was a huge help! Mike, if you ever need to know particle
physics, I'll try to return the favor -- thanks dude\ldots

UCSD would not have been nearly as much fun without my crew: Jeff, Whit, Alex,
Bourge, Frank, Toby and Diego. You guys worked and played hard and I will never
forget those time. I loved the fact that we formed a winning IM soccer team
with a bunch of grad students. Now if I can only stop yelling at the referee\ldots

I'd like to thank my advisor, Professor Avi Yagil, for giving me the opportunity to
return to school and finish what I started. I will never forget that.

Frank, I will not forget our intense moments in the trenches of this analysis
and then forgetting about it over a nice IPA. I couldn't have done this without
you dude. Furthermore, I will never forget your cross that I headed to Randy to
finish. For one rare moment, we looked like Premier League quality.

To my twin brother, Randy, I appreciate your patience and encouragement. This
was difficult undertaking but I couldn't have done it without the ``twin''
factor. Now that I'm at the end, I will actually miss our physics discussions.

Finally, to Jennifer, my wife. She has been through the whole analysis from
beginning to end. She stood by me during it all. You're love and patience was
the only way I really got through this. I can't wait to start the next chapter
with you\ldots
\end{acknowledgements}                                                         

\begin{vitapage}                                                               
\begin{vita}                                                                   
  \item[2000] B.~S. in Electrical Engineering, University of Virginia
  \item[2007] M.~S. in Physics, University of California, San Diego
  \item[2013] Ph.~D. in Physics, University of California, San Diego       
\end{vita}                                                                     
\begin{publications}                                                           
  \item Search for new physics in events with same-sign dileptons and b jets in pp collisions at $\sqrt{s} = 8$ TeV, {\it CMS Collaboration}, JHEP 1303 (2013) 037 [arXiv:1212.6194 [hep-ex]] % HCP
  %\item 2013!!!Search for new physics in events with same-sign dileptons and b jets in pp collisions at $\sqrt{s} = 8$ TeV, {\it CMS Collaboration}, JHEP 1303 (2013) 037 [arXiv:1212.6194 [hep-ex]] % 2013
\end{publications}                                                             
\end{vitapage}                                                                 
                                                                               

%% ABSTRACT
%  Doctoral dissertation abstracts should not exceed 350 words. 
%   The abstract may continue to a second page if necessary.
\begin{abstract}

A search for new physics is performed using events with isolated same-sign
leptons and jets in the final state. Results are based on the full sample
of proton-proton collisions collected from the Large Hadron Collider at a
center-of-mass energy of 8 TeV with the CMS detector and corresponding to
an integrated luminosity of 19.5 \fbinv. No excess above the standard model
background is observed and constraints on a number of new physics models are
set. 
%Information on acceptance and efficiencies is also provided so that the results can be used to confront an even broader class of new physics models.

\end{abstract}


\end{frontmatter}
